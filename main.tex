\documentclass{llncs}		%Petite taille de doc

\usepackage[francais]{babel}	%Doc fr
\usepackage[T1]{fontenc}	%Caracteres accentues
\usepackage[utf8]{inputenc}  
\usepackage{framed}
\usepackage{amsfonts}
\usepackage{amsmath}
\usepackage{algorithm}
\usepackage[noend]{algpseudocode}

\newcommand{\argmin}{\arg\!\min}
\newcommand{\argmax}{\arg\!\max}

\begin{document}
\title{Variants of Algorithm to solve the Learning Parities with Noise Problem}
\author{Thomas Bourgeat\inst{1} \and Pierre-Alain Fouque\inst{2} \and Paul Kirchner\inst{1}}
\institute{\'Ecole normale sup\'erieure, 45 rue d'Ulm, 75005 Paris, France \\
\texttt{\{Thomas.Bourgeat,Pierre-Alain.Fouque,Paul.Kirchner\}@ens.fr}
\and 
Universit\'e de Rennes and Institut Universitaire de France}

\maketitle

\begin{abstract}
In this paper, we study some variants of the BKW (Blum-Kalai-Wasserman) algorithm to solve the Learning Parities with Noise (LPN) Problem. Here, we propose to analyse these variants and propose concrete parameters for cryptosystems based on this problem. 
\end{abstract}

\section{Introduction}
The Learning Parities with Noise (LPN) problem is 
a well-known problem in learning and it is also used 
in various cryptosystems. 
Its connection with lattice problem and the average case to hard
 case reduction for some lattice has been the last years
 a very active research area and many cryptosystems have been proposed. 

%TODO
Parler de l'efficacite, LAPIN, ... 

In all the paper we use the next notations :
\begin{itemize}
\item $A.s+\nu = b$ is an instance of the LPN problem.
\item $A\in \mathcal{M}_{m\times n}(GF(2)), \nu\in \mathcal{M}_{m\times
1}(GF(2))$
\end{itemize}

\section{Description of BKW algorithm}
The main issue with the LPN is to reduce the dimension of the problem without
increase the noise too much. \cite{BKW} proposed the first sub-exponential
algorithm with a clever adaptation of gaussian pivot. Let us remind the first
step of the algorithm in a general case. Next, we are going to analyze precisely
the complexity.

\subsection{BKW}
The BKW algorithm begin with the reduction of the dimension of the problem.

\begin{framed}
\begin{algorithmic}
%BKWReduction :: Set(Sample) -> Set(Sample)  
\Function{BKWReduction}{A,b}
\Return{L}
\EndFunction
\end{algorithmic}
\end{framed}





\subsection{LF algorithm}
The main idea of the article \cite{LF} was to use a Fourier transform to recover
the secret in the last phasis of the BKW algorithm.

Indeed, you can remark that the Fourier coefficient
$$\hat{f}(x)= \sum \limits_{\forall i \in E} f(i).(-1)^{i.x}$$
is highly correlated to the score: $$score(x)=\sum\limits_{\forall r_i\in
Row(A)}r_i.x+b(i)$$ The first time, we are in $({-1,1},*)$, the second time we
are in $(\mathbb{Z}/2\mathbb{Z},\oplus)$.Moreover, in our context of LPN we have
$s=\argmin\limits_{x \in E} score(x)$. In other words, we want to say $s = \argmax\limits_{x\in
E} \hat{f}(x)$.

So, if we know the greater coefficient of the Fourier tranforms of f, we solve
the LPN. Actually, there is a deeper reason for this correlation that you can find in any
lesson on the representation of finite abelian groups. 

\subsection{Algo de Paul sur eprint}



\section{Description of the Algorithm}

decrire le nouvel algo en indiquant les differentes phase grace a un dessin

\subsection{Complexity}

Complexity en fonction de la sparse WH

\section{Sparse Walsh-Hadamard}
Let $f:(\mathbb{Z}/2.\mathbb{Z})^n \rightarrow \mathbb{Z}/2.\mathbb{Z}$.

\subsection{Description}
We know how to compute a fast Walsh-Hadamard transform.
However, to solve the sparse-LPN, we don't need the all spectrum of f. 
A natural question is to wonder if we can adapt the algorithm to compute only
some coefficients of $\hat{f}$. 

The next algorithm generates the list of $(s,score(s))$ for all $s$ potential secret of
hamming weight less or equal to p.p is a parameter of the algorithm.
\begin{framed}
\begin{algorithmic}
%HW :: Set(Sample) -> List(secrets,scores)  
\State p parameter of sparsity
\Function{WH}{Samples,n}
\If{$n=0$} \Return{$[(\epsilon,\sum\limits_{(s,v) \in \text{Samples}} 2v-1)]$}  \EndIf
\State $L_0 \gets \{x | 0.x \in \text{Samples} \}$
\State $L_1 \gets \{x | 1.x \in \text{Samples} \}$
\State $s_0 \gets WH(L_0,n-1)$ 
\State $s_1 \gets WH(L_1,n-1)$
\State$ L \gets []$
\ForAll{ $(s,score_0) \in s_0,(s,score_1) \in s_1 $ } 
  \State $ L\gets L:(0.s, score_0 + score_1)    $
  \If{$weight(1.s) \leq p $} $L\gets L: (1.s ,  score_0 - score_1)  $ \EndIf 
\EndFor
\Return{L}
\EndFunction
\end{algorithmic}
\end{framed}

\subsection{Complexity}

\section{Tables}

\section{Experiments}

\section{Conclusion}

%% References
\bibliographystyle{inc/splncs_srt.bst}
%\bibliographystyle{alpha}
%\bibliographystyle{apalike}
\bibliography{bibliography}

%% Appendix
%\appendix
%\input{appendix.tex}
 



\end{document}
